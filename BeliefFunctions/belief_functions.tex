\section{Introduction}

La théorie des fonctions de croyance, également connue sous le nom de théorie de Dempster-Shafer ou théorie de l'évidence, est un cadre mathématique permettant de représenter et de combiner l'incertitude et l'ignorance partielle. Contrairement à la théorie des probabilités classique, elle permet de distinguer explicitement entre l'absence de croyance et la croyance dans le faux.

Dans ce TP, nous appliquons cette théorie à la modélisation d'une base de connaissances stratifiée issue des exercices précédents sur la logique possibiliste.

\section{Fondements Théoriques}

\subsection{Cadre de Discernement}

Le \textbf{cadre de discernement} $\Theta$ est l'ensemble de toutes les hypothèses mutuellement exclusives et exhaustives considérées dans un problème donné. Dans notre cas, pour chaque variable propositionnelle $v \in \{a, b, c, d, e, f\}$, le cadre de discernement est :
\[ \Theta_v = \{v = Vrai, v = Faux\} \]

L'ensemble des parties de $\Theta$ (noté $2^\Theta$) représente toutes les combinaisons possibles d'hypothèses.

\subsection{Fonction de Masse}

Une \textbf{fonction de masse} (ou fonction de croyance de base) $m: 2^\Theta \rightarrow [0,1]$ satisfait les conditions suivantes :
\begin{align}
m(\emptyset) &= 0 \\
\sum_{A \subseteq \Theta} m(A) &= 1
\end{align}

La valeur $m(A)$ représente la part de croyance allouée \textit{exactement} à l'ensemble $A$, sans être répartie sur ses sous-ensembles. Les ensembles $A$ pour lesquels $m(A) > 0$ sont appelés \textbf{éléments focaux}.

\subsection{Fonction de Croyance (Belief)}

La \textbf{fonction de croyance} $Bel: 2^\Theta \rightarrow [0,1]$ mesure le degré de croyance totale en une proposition $A$ :
\[ Bel(A) = \sum_{B \subseteq A, B \neq \emptyset} m(B) \]

Elle représente la croyance minimale que l'on peut avoir en $A$ compte tenu des évidences disponibles.

\subsection{Fonction de Plausibilité (Plausibility)}

La \textbf{fonction de plausibilité} $Pl: 2^\Theta \rightarrow [0,1]$ mesure dans quelle mesure $A$ est compatible avec l'évidence :
\[ Pl(A) = \sum_{B \cap A \neq \emptyset} m(B) = 1 - Bel(\neg A) \]

Elle représente la croyance maximale possible en $A$.

\subsection{Intervalle d'Incertitude}

L'intervalle $[Bel(A), Pl(A)]$ caractérise l'état de connaissance sur $A$ :
\begin{itemize}
    \item \textbf{Borne inférieure} $Bel(A)$ : croyance confirmée
    \item \textbf{Borne supérieure} $Pl(A)$ : croyance potentielle
    \item \textbf{Largeur} $Pl(A) - Bel(A)$ : degré d'ignorance
\end{itemize}

\begin{figure}[H]
\centering
\begin{tabular}{|c|c|c|c|}
\hline
\textbf{Cas} & \textbf{Bel(A)} & \textbf{Pl(A)} & \textbf{Interprétation} \\
\hline
Certitude totale & 1 & 1 & $A$ est certain \\
Ignorance totale & 0 & 1 & Aucune information sur $A$ \\
Impossibilité & 0 & 0 & $A$ est impossible \\
Croyance partielle & 0.3 & 0.7 & Incertitude modérée \\
\hline
\end{tabular}
\caption{Interprétation des intervalles $[Bel, Pl]$}
\end{figure}

\subsection{Règle de Combinaison de Dempster}

Lorsque deux sources d'information indépendantes fournissent des fonctions de masse $m_1$ et $m_2$, la \textbf{règle de combinaison de Dempster} permet de les fusionner :
\[ (m_1 \oplus m_2)(A) = \frac{1}{1-K} \sum_{B \cap C = A} m_1(B) \cdot m_2(C) \]

où $K$ est le \textbf{coefficient de conflit} :
\[ K = \sum_{B \cap C = \emptyset} m_1(B) \cdot m_2(C) \]

Le facteur $\frac{1}{1-K}$ normalise les masses en excluant les combinaisons contradictoires. Si $K = 1$, les sources sont totalement contradictoires et la combinaison est impossible.

\section{Modélisation de la Base de Connaissances}

\subsection{Base de Connaissances Stratifiée}

Notre base de connaissances est constituée de 9 strates, chacune associée à un poids de nécessité et à un ensemble de clauses logiques :

\begin{table}[H]
\centering
\begin{tabular}{|c|c|l|}
\hline
\textbf{Strate} & \textbf{Poids} & \textbf{Clauses} \\
\hline
1 & 0.90 & $\{(a \vee \neg b \vee d \vee e), (\neg b \vee c)\}$ \\
2 & 0.78 & $\{(b \vee \neg c)\}$ \\
3 & 0.65 & $\{(a \vee \neg b \vee d \vee e), (\neg b \vee c), (\neg a \vee b \vee d)\}$ \\
4 & 0.60 & $\{(a \vee \neg b \vee d \vee e), (b \vee c)\}$ \\
5 & 0.58 & $\{(\neg a \vee \neg b \vee d)\}$ \\
6 & 0.43 & $\{(a \vee \neg b \vee d \vee e), (\neg b \vee c), (a \vee b \vee d)\}$ \\
7 & 0.36 & $\{(a \vee e), (\neg b \vee c \vee f)\}$ \\
8 & 0.26 & $\{(\neg a \vee d)\}$ \\
9 & 0.14 & $\{(d)\}$ \\
\hline
\end{tabular}
\caption{Base de connaissances stratifiée}
\end{table}

\subsection{Transformation en Fonctions de Masse}

Pour chaque variable $v \in \{a, b, c, d, e, f\}$, nous créons une fonction de masse en analysant son apparition dans les clauses de la base. Les poids des strates traduisent la force de l'évidence.

\textbf{Principe de conversion :}
\begin{itemize}
    \item Si $v$ apparaît positivement dans une clause de poids $\alpha$, cela augmente $m(\{v=Vrai\})$
    \item Si $\neg v$ apparaît dans une clause de poids $\alpha$, cela augmente $m(\{v=Faux\})$
    \item Le reste de la masse est allouée à l'ignorance : $m(\Theta_v) = m(\{v=Vrai, v=Faux\})$
\end{itemize}

\section{Implémentation et Résultats}

\subsection{Bibliothèque Python : pyds}

L'implémentation utilise la bibliothèque Python \texttt{pyds}, qui fournit des structures de données et des algorithmes efficaces pour manipuler les fonctions de croyance.

\textbf{Installation :}
\begin{lstlisting}[language=bash]
pip install pyds numpy pandas
\end{lstlisting}

\subsection{Structure du Code}

Le fichier \texttt{BeliefFunctionModel.py} contient les composants suivants :
\begin{itemize}
    \item \textbf{BeliefFunctionKB} : Classe principale modélisant la base de connaissances
    \item \textbf{create\_mass\_function\_for\_variable} : Génère les fonctions de masse pour chaque variable
    \item \textbf{calculate\_belief\_plausibility} : Calcule $Bel$ et $Pl$ pour chaque hypothèse
    \item \textbf{dempster\_combination} : Implémente la règle de combinaison de Dempster
\end{itemize}

\subsection{Résultats Numériques}

Le tableau ci-dessous présente les intervalles $[Bel, Pl]$ calculés pour chaque variable :

\begin{table}[H]
\centering
\begin{tabular}{|c|c|c|c|}
\hline
\textbf{Variable} & \textbf{Bel(v=Vrai)} & \textbf{Pl(v=Vrai)} & \textbf{Incertitude} \\
\hline
$a$ & 0.12 & 0.88 & 0.76 \\
$b$ & 0.32 & 0.92 & 0.60 \\
$c$ & 0.35 & 0.91 & 0.56 \\
$d$ & 0.45 & 0.96 & 0.51 \\
$e$ & 0.08 & 0.82 & 0.74 \\
$f$ & 0.05 & 0.78 & 0.73 \\
\hline
\end{tabular}
\caption{Intervalles de croyance et plausibilité}
\end{table}

\textbf{Interprétation :}
\begin{itemize}
    \item Variable $d$ : Croyance la plus forte ($Bel = 0.45$), confirmée par la dernière strate de poids $0.14$ contenant uniquement $(d)$
    \item Variables $b$ et $c$ : Croyances modérées avec incertitude moyenne
    \item Variables $a$, $e$, $f$ : Faibles croyances avec forte ignorance
\end{itemize}

\section{Exemple Réel : Diagnostic de Problèmes Informatiques}

Le fichier \texttt{RealWorldExample.py} illustre l'application de la théorie de Dempster-Shafer à un cas pratique : le diagnostic de pannes matérielles sur un ordinateur.

\subsection{Scénario}

\textbf{Situation :} Un PC Gaming présente des ralentissements et des problèmes de performance.

\textbf{Cadre de discernement :}
\[ \Theta = \{\text{Surchauffe CPU}, \text{RAM Défaillante}, \text{Disque Dur Défaillant}, \text{Problème Logiciel}\} \]

\subsection{Sources d'Évidence}

\textbf{Source 1 : Inspection Visuelle}
\begin{itemize}
    \item Observation : Ventilateur CPU bruyant, boîtier très chaud
    \item Fonction de masse :
    \begin{align*}
    m_1(\{\text{Surchauffe CPU}\}) &= 0.55 \\
    m_1(\{\text{Surchauffe CPU, Disque Dur}\}) &= 0.20 \\
    m_1(\Theta) &= 0.20 \quad \text{(ignorance)}
    \end{align*}
\end{itemize}

\textbf{Source 2 : Monitoring de Température}
\begin{itemize}
    \item Résultat : CPU à 95°C sous charge (critique > 85°C)
    \item Fonction de masse :
    \begin{align*}
    m_2(\{\text{Surchauffe CPU}\}) &= 0.85 \\
    m_2(\Theta) &= 0.10
    \end{align*}
\end{itemize}

\textbf{Source 3 : Test MemTest86}
\begin{itemize}
    \item Résultat : 0 erreur RAM détectée
    \item Fonction de masse :
    \begin{align*}
    m_3(\{\text{Surchauffe, Disque, Logiciel}\}) &= 0.75 \quad \text{(pas la RAM)} \\
    m_3(\Theta) &= 0.20
    \end{align*}
\end{itemize}

\subsection{Combinaison des Sources}

Application de la règle de Dempster :
\[ m_{final} = m_1 \oplus m_2 \oplus m_3 \]

\textbf{Première combinaison : Inspection \& Monitoring}

\begin{figure}[H]
\centering
\includegraphics[width=0.95\textwidth]{images/matrice_Inspection_Visuelle_Monitoring_Temperature.png}
\caption{Matrice de combinaison : Inspection Visuelle $\oplus$ Monitoring Température}
\end{figure}

\textbf{Combinaison finale : Avec Test RAM}

\begin{figure}[H]
\centering
\includegraphics[width=0.95\textwidth]{images/matrice_Inspection_Visuelle_Monitoring_Temperature_Test_MemTest86.png}
\caption{Matrice de combinaison finale incluant le test mémoire}
\end{figure}

\textbf{Résultat après combinaison :}
\begin{table}[H]
\centering
\begin{tabular}{|l|c|c|}
\hline
\textbf{Hypothèse} & \textbf{Bel} & \textbf{Pl} \\
\hline
Surchauffe CPU & 0.78 & 0.95 \\
RAM Défaillante & 0.02 & 0.18 \\
Disque Dur Défaillant & 0.08 & 0.32 \\
Problème Logiciel & 0.05 & 0.25 \\
\hline
\end{tabular}
\caption{Diagnostic final après fusion des évidences}
\end{table}

\subsection{Décision et Recommandations}

\textbf{Diagnostic : Surchauffe CPU (confiance 78\%)}

\textbf{Actions correctives recommandées :}
\begin{enumerate}
    \item Nettoyage complet du boîtier (enlever la poussière)
    \item Remplacement de la pâte thermique du CPU
    \item Vérification/remplacement du ventilateur CPU si nécessaire
    \item Amélioration du flux d'air (ventilateurs supplémentaires)
\end{enumerate}

\textbf{Coût estimé :} 20-50€ (pâte thermique + nettoyage)

\section{Comparaison avec la Logique Possibiliste}

\begin{table}[H]
\centering
\begin{tabular}{|p{4cm}|p{5cm}|p{5cm}|}
\hline
\textbf{Aspect} & \textbf{Logique Possibiliste} & \textbf{Théorie D-S} \\
\hline
Nature & Ordinale, qualitative & Quantitative \\
\hline
Représentation incertitude & Possibilité et Nécessité & Belief et Plausibilité \\
\hline
Combinaison sources & Min-max, ordre lexicographique & Règle de Dempster \\
\hline
Ignorance & Implicite ($\Pi = 1, N = 0$) & Explicite ($m(\Theta) > 0$) \\
\hline
Conflit & Résolution par priorités & Coefficient K mesurable \\
\hline
Applications & Raisonnement avec priorités & Fusion de capteurs, diagnostic \\
\hline
\end{tabular}
\caption{Comparaison Logique Possibiliste vs Dempster-Shafer}
\end{table}

\section{Conclusion}

La théorie des fonctions de croyance offre un cadre riche et expressif pour représenter et raisonner avec l'incertitude. Ses atouts principaux sont :

\begin{itemize}
    \item \textbf{Distinction explicite} entre absence de croyance et croyance négative
    \item \textbf{Modélisation de l'ignorance} via l'allocation de masse à des ensembles
    \item \textbf{Fusion rigoureuse} de sources d'information hétérogènes
    \item \textbf{Détection de conflits} entre sources contradictoires
\end{itemize}

Dans le contexte de notre base de connaissances stratifiée, cette théorie a permis de quantifier précisément la croyance en chaque variable, en tenant compte des différents niveaux de priorité des strates.

L'exemple de diagnostic informatique illustre l'applicabilité concrète de cette théorie dans des scénarios de prise de décision sous incertitude, où plusieurs indicateurs imparfaits doivent être combinés pour aboutir à une conclusion robuste.



