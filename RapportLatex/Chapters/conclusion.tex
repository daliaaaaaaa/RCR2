\section*{Conclusion}

Ce travail a illustré deux approches complémentaires du raisonnement sous incertitude appliquées à des problématiques informatiques concrètes.

La logique floue, à travers le contrôleur de type Mamdani, a permis de modéliser l'allocation de ressources CPU de manière progressive et intuitive, en s'appuyant sur des règles linguistiques traduisant l'expertise humaine. Cette approche offre une alternative flexible aux méthodes d'ordonnancement classiques basées sur des seuils rigides.

La logique possibiliste, combinée aux solveurs SAT, a démontré sa capacité à gérer des connaissances stratifiées par niveau de priorité. L'utilisation de la méthode de dichotomie avec le solveur Glucose 4 a permis d'identifier efficacement les niveaux de nécessité des propositions, illustrant ainsi la puissance de la fusion entre raisonnement qualitatif et techniques algorithmiques modernes.

Ces deux paradigmes montrent que l'intégration de l'incertitude et de l'imprécision dans les systèmes informatiques peut améliorer significativement leur adaptabilité et leur performance face à des environnements complexes et dynamiques.